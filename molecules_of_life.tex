\documentclass{article}

\author{Migdal}
\date{27/10/2016}
\title{Proteins, Polysacharydes, Lipids and Nucleic Acid - Overview}

\usepackage{xymtexpdf}
\usepackage{xcolor}
\usepackage{graphicx}

\wedgehashedwedge%global declaration

\begin{document}
\maketitle

%%%%%%%%%%%%%%%%%%%%%%%%%%%%%%%%%%%%%%%%%%%%%%%%%%%%%%%%%%%%%%%%%%%%%%%%%%%%%%%%%%%%%%

\section{Lipids}
We call \textbf{lipids} natural organic molecules with little solubility in water,
isolated from cells and tissues by extraction with non-polar solvents. Examples of
lipids are: fats, oils, waxes, hormones and most of non-protein cell membrane
elements. \textit{Note that lipids definition is based on physical property,
solubility and not structure !}
\\
Lipids can be divided into two basic classes.
\paragraph{Fats and waxes}
are lipids that contain ester bonds and cant be subjected to hydrolysis.\\
\dtrigonal{0==C;1D==O;2=={OR'};3=={R}} Ester bond\\
The general structure of animal fat is as follows:\\

\hspace{7cm}\tetrahedral{0==CHO;%%1
1==\tetrahedral{0==CH$_2$O;3==(yl);%%4
		4==\tetrahedral{0==C;1D==O;2==(yl);4==R}};%%3
3==\tetrahedral{0==CH$_2$O;1==(yl);%%4
		4==\tetrahedral{0==C;1D==O;2==(yl);4==R"}};%%4
4==\tetrahedral{0==C;1D==O;2==(yl);4==R'}}<300,0,300,100>
\newpage
\paragraph{Steroids}
They have no ester bonds and can't be subjected to hydrolysis, as an example consider 
cholesterol:\\
\cholestanE[e]{3B==HO}
Their structure is based on four cyclic system, three six membered rings (in chair
like conformation) and one five membered ring. In humans most of steroids play role
of hormones, regulatory signals (examples: testosterone, estron).

\subsection{Waxes, fats and oils}
\paragraph{Waxes}
are mixtures of long chained fatty acid esters and long chained alcohols. For
example heksadekanin triakontylu, main ingredient of bee wax, consist of alcohol
C$_{30}$ and fatty acid C$_{16}$ ester. Protector layers on fruits and vegetables have
very similar structure.

\paragraph{Fats and oils} are most common lipids. They are quite different when it
comes to physical properties, like melting point, but they share very similar
structure. In a chemical sense they are triacylogricerols, triesters of glycerin
with tree long chained carboxylic acids. Those acids doesn't have to be the same.
Composition of those fatty acids is very important, as they have big impact on
melting points. Generally the more unsaturated fatty acids in a fat the lower is
melting point temperature. This results from differences in structure, saturated
acids have very regular structure and can be easily packed into crystal network.

\subsection{Phospholipids}
Phospholipids are esters of phosphoric acid, H$_3$PO$_4$. There can be distinguished
two main types of phospholipids: \textit{glicerophospholipids and sfingomielins}.
\paragraph{Glicerophospholipids} are closely related to fats and oils, in a
structural sense. They are build out of glycerin skeleton connected through ester
bonds with two molecules of fatty acids and one phosphate acid. Commonly phosphate
group is also connected through another ester bond with aminoalkohol
like choline [(CH$_3$)$_3$NCH$_2$CH$_2$OH]$^+$ or etylamine H$_2$NCH$_2$CH$_2$OH. The
most important glycerolophospholipids are lecytines and cefalines. 

\vspace{5mm}
\tetrahedral{0==CH;%%1
1==\tetrahedral{0==CH$_2$O;3==(yl);%%4
		4==\tetrahedral{0==C;1D==O;2==(yl);4==R'}};%%3
3==\tetrahedral{0==CH$_2$O;1==(yl);%%4
		4==\tetrahedral{0==P;1D==O;2==(yl);3==O$^-$;4==OR"}};%%2
2==\tetrahedral{0==C;1D==O;2==RO;4==(yl)}}<300,100,300,100>
\vspace{5mm}
\begin{flushright}
Where R": CH$_2$CH$_2$N(CH$_3$)$_3$ for lecytyne,\\
CH$_2$CH$_2$NH$_3$ for cefaline\\
\end{flushright}

Glicerolophospholipides are main component of cell membranes (~40\%). 
Glicerolophospholipides have long non polar tails connected with polar head (phosphate 
group), thank to this they can aggregate into bi layer membrane, which is effective 
barrier against water, ions and other polar molecules outside and inside the cell.

\paragraph{Sfingomielins} are other group of phospholipids. This are compounds
with sfingozine or dihidroksyamin skeleton. They are ingredients of animal and plats
cell membranes. They are most abundant in brain and nerve tissue.

\subsection{Others}
It should be noted that there are many more subgroups of lipis. One example are
terpenoids - found in ether oils.
\section{Nucleic Acids}
\end{document}

